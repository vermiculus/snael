\documentclass{beamer}

\usepackage{mypostersetup}

\begin{document}
\begin{frame}[t]{}
  \vskip -8pt
  \begin{beamercolorbox}{}
    \maketitle
  \end{beamercolorbox}
  \vskip 10ex

  \begin{columns}
    \begin{column}{.45\textwidth}
      \begin{block}{Introduction}
        %What has currently been done, what is it good for, what are
        %you trying to do, start with the general, going into
        %specifics

        Computer science is more than just programming software for
        industry; a great portion of `research in computer science' is
        actually that of algorithms, of complexity theory, and of
        human-computer interaction.

        What is \ac{nlp}?  \Ac{nlp} is an incredibly broad topic with
        tens of subfields, but here are some highlights for the
        current research:

        \begin{description}[\Ac{ner}\acreset{ner}]
        \item[Part-of-Speech Tagging] tries to recognize the
          syntactical positions of words in a sentence
        \item[\Ac{ner}] identifies nouns in a text and categorizes
          them into groups (e.g.~location, organization, person, \dots)
        \item[Coreference resolution] deals with aliases of entities
          (both proper and common) in a text
        \end{description}

        Why study \ac{nlp}?  \Ac{nlp} is a significant part of
        human-computer interaction.

        \begin{itemize}
        \item Automated telephone dispatchers (i.e. voice-jail) try to
          hear,\footnote{speech recognition}
          understand,\footnote{\ac{nlp}} and resolve\footnote{\ac{nlp}
            $+$ natural language \emph{generation}} your spoken
          request
        \end{itemize}

      \end{block}

% ILLUSTRATION:
% Natural Language Processing
% -> Theoretical Computer Science {Formal Languages, Context-free
%    Grammars
% -> Text Processing
% -> Machine Learning
% -> Artificial Intelligence
% -> Human-Computer Interaction
% -> Linguistics

      \begin{block}{Objective}
        \begin{itemize}
        \item Read text
        \item Recognize names
        \item Relate aliases
        \item Reveal co-occurrences
        \end{itemize}
      \end{block}

      \begin{block}{Methods}
        Source Listings, what a name is defined as
      \end{block}
    \end{column}

    \begin{column}{.45\textwidth}
      \begin{block}{Results}
        I am \emph{not} qualified to draw conclusions.  Ha!
      \end{block}

      \begin{block}{Discussion} 
        The meaning of results, the future of the topic, error covering
      \end{block}

      \begin{block}{Conclusions}
        a re-hash of the second part of discussion?
      \end{block}

      \begin{block}{Acknowledgments}
        Thanks Dr.~P\'adraig~Cunningham
      \end{block}
    \end{column}
  \end{columns}
\end{frame}
\end{document}

%%% Local Variables:
%%% TeX-master: t
%%% TeX-PDF-mode: t
%%% End: