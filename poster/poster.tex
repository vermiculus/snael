\documentclass{beamer}
\usepackage[english]{babel}

\usepackage[orientation=portrait,size=a0]{beamerposter}
\mode<presentation>{%
  \usetheme{Frankfurt}%
}

\newcommand{\llogo}{\includegraphics[height=4cm]{logo}}
\newcommand{\rlogo}{\includegraphics[height=2cm]{logo}}

\setbeamerfont{block title}{size=\LARGE}

\title[S. N. A. E. L.]{%
\parbox{.25\textwidth}{\centering\rule[-.3\baselineskip]{0pt}{5cm}\llogo}%
\parbox{.5\textwidth}{\centering\LARGE Social Network Analysis in English Literature}%
\parbox{.25\textwidth}{\centering\rlogo}%
}
\author{\Large Sean Allred}
\institute{\Large University College Dublin, Department of Computer Science and Informatics%
%\\\large Scoil na R\'iomheola\'iochta agus Faisn\'eis\'iochta, An Col\'aiste Ollscoile, Baile \'Atha Cliath
}
\date{\Large 26 April 2013}

\begin{document}
\begin{frame}[t]{}
  \vskip -1ex
  \begin{beamercolorbox}{}
    \maketitle
  \end{beamercolorbox}
  \vskip 10ex
  \begin{columns}
    \begin{column}{.45\textwidth}
      \begin{block}{Introduction} % What has currently been done, what is it good for, what are you trying to do, start with the general, going into specifics
        Why, hello there!  \LaTeX{} is a tool for professional
        typesetting.  It is used by academics, researchers, and
        professionals \emph{all over the world}.  It is written in
        plain text in the \LaTeX{} format.  Like HTML, \LaTeX{} is a
        \textsl{markup language}; while Turing complete for all my
        geeks out there, it is designed to be simple to understand
        even in its plain-text form.
      \end{block}
      \begin{block}{Objective}
        doodaa
      \end{block}
    \end{column}
    \begin{column}{.45\textwidth}
      \begin{block}{Methods} % Source Listings, what a name is defined as
        Writing \LaTeX{} is easy: all you need is an editor (any text
        editor will do) and what we call a \TeX{} distribution.  A
        \TeX{} distribution is just a fancy way of saying `everything
        you need to produce a document with \LaTeX{}.  This usually
        includes the main programs (\texttt{tex}, \texttt{latex},
        \texttt{pdftex}, \texttt{pdflatex}, etc.) and a slew of other
        helpful, supplementary programs (\texttt{texdoc},
        \texttt{texhash}, etc.).  This may seem like a lot, but the
        only program you will ever have any dealings with on a regular
        basis is in fact \texttt{pdflatex},\footnote{\texttt{tex} and
          \texttt{latex} turn these source files into what is called a
          DVI, an even more portable format.} the program that will
        take a properly-formatted \LaTeX{} source file and turn it
        into a PDF.
      \end{block}
      \begin{block}{Results}
        I am \emph{not} qualified to draw conclusions.  Ha!
      \end{block}
      \begin{block}{Discussion} % The meaning of results, the future of the topic, error covering
        I am \emph{not} qualified to draw conclusions.  Ha!
      \end{block}
      \begin{block}{Conclusions} % a re-hash of the second part of discussion?
        I am \emph{not} qualified to draw conclusions.  Ha!
      \end{block}
      \begin{block}{Acknowledgments} % a re-hash of the second part of discussion?
        I am \emph{not} qualified to draw conclusions.  Ha!
      \end{block}
    \end{column}
  \end{columns}
\end{frame}
\end{document}

%%% Local Variables:
%%% TeX-master: t
%%% TeX-PDF-mode: t
%%% End: