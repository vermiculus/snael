\documentclass{beamer}
\usepackage[english]{babel}

\makeatletter
\newcommand\HUGE{\@setfontsize\Huge{96}{115}}
\makeatother

\usepackage[orientation=portrait,size=a0]{beamerposter}
\mode<presentation>{%
  \usetheme{Frankfurt}%
}

\hypersetup{
    pdftitle={Social Network Analysis of English Literature},
    pdfauthor={Sean Allred},
    pdfsubject={Natural Language Processing},
    pdfkeywords={nlp} {computer} {science} {conference} {poster},
    colorlinks=true,
    linkcolor=black,
    citecolor=black,
    filecolor=black,
    urlcolor=black}

\newcommand{\llogo}{\includegraphics[height=11.88cm]{ucd}}
\newcommand{\rlogo}{\includegraphics[height=11.88cm]{ucd}}

\setbeamerfont{block title}{size=\LARGE}
\setbeamersize{text margin left = -5pt, text margin right = -5pt}

\addtobeamertemplate{block end}{}{\vskip 10ex plus 1ex minus 1ex}

\title[S. N. A. E. L.]{%
\texorpdfstring{\parbox{.25\textwidth}{\centering\llogo}%
\parbox{.5\textwidth}{\centering\HUGE Social Network Analysis of English Literature}%
\parbox{.25\textwidth}{\centering\rlogo}}{Real Title}}
\author{\Huge Sean Allred}
\institute{\Large University College Dublin\\\large Department of Computer Science and Informatics%
%\\\large Scoil na R\'iomheola\'iochta agus Faisn\'eis\'iochta, An Col\'aiste Ollscoile, Baile \'Atha Cliath
}
\date{\Large 26 April 2013}

% \title[SNAIL]{\HUGE Social Network Analysis of English Literature}
% \makeatletter
% \addtobeamertemplate{title page}{%
%     \let\beamerorig@inserttitle\inserttitle%
%     \renewcommand{\inserttitle}{%
%        \parbox{0.20\textwidth}{\centering\llogo}
%        \parbox{0.5 \textwidth}{\centering\beamerorig@inserttitle}
%        \parbox{0.20\textwidth}{\centering\rlogo}
%     }
% }{}
% \makeatother


\usepackage{acro}
\DeclareAcronym{nlp}{
  short= NLP,
  long= Natural Language Processing}
\DeclareAcronym{nltk}{
  short= NLTK,
  long= Natural Language Processing Toolkit}

\newcommand{\topic}[1]{\textsl{#1}}

\begin{document}
\begin{frame}[t]{}
  \vskip -8pt
  \begin{beamercolorbox}{}
    \maketitle
  \end{beamercolorbox}
  \vskip 10ex

  \begin{columns}
    \begin{column}{.45\textwidth}
      \begin{block}{Introduction}
        %What has currently been done, what is it good for, what are
        %you trying to do, start with the general, going into specifics
        Computer science is a \emph{very} broad topic, covering
        everything from the stereotypical to the incredibly novel, and
        the entire study is growing exponentially\footnote{Ha!  Puns!}
        fast.  Those which were once chased as a pipe-dreams and
        featured in science fictions has, in many cases, become
        realized in an industry-grade implementation.  However, not
        all pipe-dreams have become reality so quickly.

        \ac{nlp} is one such growing subfield of computer science that
        would enable a computer, through some algorithm, to ingest and
        interact with human language.  This is distinct from speech
        recognition, in which the goal is to transform spoken language
        into text, but not necessarily understand it.

        A further (major) subfield of \ac{nlp} is \topic{named entity
          recognition}
      \end{block}

% ILLUSTRATION:
% Natural Language Processing
% -> Theoretical Computer Science {Formal Languages, Context-free
%    Grammars
% -> Text Processing
% -> Machine Learning
% -> Artificial Intelligence
% -> Human-Computer Interaction
% -> Linguistics

      \begin{block}{Objective}
        doodaa
      \end{block}

      \begin{block}{Methods}
        Source Listings, what a name is defined as
      \end{block}
    \end{column}

    \begin{column}{.45\textwidth}
      \begin{block}{Results}
        I am \emph{not} qualified to draw conclusions.  Ha!
      \end{block}

      \begin{block}{Discussion} 
        The meaning of results, the future of the topic, error covering
      \end{block}

      \begin{block}{Conclusions}
        a re-hash of the second part of discussion?
      \end{block}

      \begin{block}{Acknowledgments}
        Thanks Dr.~P\'adraig~Cunningham
      \end{block}
    \end{column}
  \end{columns}
\end{frame}
\end{document}

%%% Local Variables:
%%% TeX-master: t
%%% TeX-PDF-mode: t
%%% End: