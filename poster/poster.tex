\documentclass{beamer}

\usepackage{mypostersetup}
\usepackage{listings,textcomp,xspace}

\beamertemplategridbackground[29mm]

\addbibresource{../../refs/nlp.bib}

\newcommand{\Gephi}{\textit{Gephi}\xspace}

\begin{document}

\begin{frame}[t,fragile]{}
  \vskip -8pt
  \begin{beamercolorbox}{}
    \maketitle
  \end{beamercolorbox}
  \vskip 10ex

  \begin{columns}
    \begin{column}{.45\textwidth}
      \begin{block}{Introduction}
        The current research focuses on a specific topic in computer
        science called \ac{nlp}, briefly described as the study and
        practice of interfacing human languages with that of machines.

        \Ac{nlp} is an incredibly broad topic with tens of subfields,
        but here are some highlights that are especially relevant to
        the current research:

        \begin{description}[\Ac{ner}\acreset{ner}]
        \item[Part-of-Speech Tagging] tries to recognize the
          syntactical positions of words in a sentence
        \item[\Ac{ner}] identifies nouns in a text and categorizes
          them into groups (e.g.~location, organization, person, \dots)
        \item[Coreference resolution] deals with aliases of entities
          (both proper and common) in a text
        \end{description}

        There are \emph{incredible} applications of \ac{nlp}, and it
        is a significant part of human-computer interaction.  The
        fantastic computers of the \textit{Star Trek} series portray
        (contrived) displays of near-perfect \ac{nlp}.  While we are a
        \emph{long} way off from this level of sophistication, we have
        still come a long way from the pipe-dreams of the twentieth
        century.

        \begin{itemize}
        \item Automated telephone dispatchers try to hear, understand,
          and resolve your spoken request through speech recognition
          and natural language processing/generation.
        \item The ever popular \textit{Siri} program of Apple's iOS
          uses natural language processing coupled with
          \textit{machine learning}, gathering statistical data from
          thousands upon thousands of users \parencite{lohr12:big-data-age}
        \item The \textit{Google} web search engine uses natural
          language processing to extract keywords and topics from the
          web pages it indexes \parencite{ganchev12:search-logs-query-tagging}
        \item The \textit{Wolfram Alpha} project (of Wolfram Research)
          provides a natural language interface to a vast wealth of
          information \parencite{press:wolfram09:wa-launch}
        \end{itemize}
      \end{block}

% ILLUSTRATION:
% Natural Language Processing
% -> Theoretical Computer Science {Formal Languages, Context-free
%    Grammars
% -> Text Processing
% -> Machine Learning
% -> Artificial Intelligence
% -> Human-Computer Interaction
% -> Linguistics

      \begin{block}{Objective}
        The goal of this research was to investigate the potential of
        using Python in conjuction with the
        \ac{nltk} \parencite{bird09:nltk} to produce an easy-to-read
        and extensible framework in which to analyse the social
        networks between characters within English literature.

        The end product is a text file containing three fields:
        \vspace{1ex plus 0.1ex minus 0.1ex}
        
        \begin{lstlisting}[frame=shadowbox]
          Character A, Character B, Number-of-Cooccurrences
        \end{lstlisting}


        \begin{itemize}
        \item character
        \item other character
        \item number of co-occurrences
        \end{itemize}

        This file is to be processed by a separate visualization and
        analytical program called \Gephi.

        For example, processing a work such as \textit{Les
          Mis\'erables} may produce the following (example) output:

        \vspace{1ex plus 0.1ex minus 0.1ex}
        
        \begin{lstlisting}[frame=shadowbox]
          Valjean, Cosette, 2000
          Valjean, Marius,  1200
          Valjean, Javert,   650
          Fantine, Javert,   250
          ......................
        \end{lstlisting}

        which would produce something similar to following:

        \includegraphics[width=.9\textwidth]{lesm-net}
      \end{block}
    \end{column}

    \begin{column}{.45\textwidth}

      \begin{block}{Methods}

        \begin{enumerate}
        \item To begin the project, the \ac{nltk} (written by
          \cite{bird09:nltk}) was used to tag each sentence from the
          entire text, a process that takes a comparatively significant
          amount of time.
          
          In addition to its normal procedures, I further defined to the
          \ac{nltk} what it should recognize as a \emph{name} by means
          of a \emph{regular expression}, a pattern by which to
          recognize the name.
          \medskip
          \begin{lstlisting}[language=Python,
                             basicstyle=\ttfamily,
                             frame=shadowbox]
            grammar = r'NAME: {<NNP>+(<DT>?<NNP>+)?}'
          \end{lstlisting}
          This grammar defines a \lstinline!NAME! to be one or more
          proper nouns optionally followed by an optional determiner and
          one or more other proper nouns.
          Simply, this grammar will recognize
          \medskip
          \begin{lstlisting}[frame=shadowbox]
            John Adams
            Alexander the Great
            William
          \end{lstlisting}
          all as \lstinline!NAME!s.

        \item Once everything is tagged, the names are stripped from
          the text and a simple algorithm is written to combine two
          names:
          \begin{enumerate}
          \item If they are the exact same name,
          \item If one name contains the other (e.g. \textsl{Alan}
            vs. \textsl{Alan Turing}), or
          \item If the two names are sufficiently statistically
            similar (using $n$-grams), they are the same.
          \end{enumerate}
        \item Now that we have a rough idea for how a person is
          referred to, we scan the text for co-references and output
          to a text file for processing by \Gephi.
        \end{enumerate}
      \end{block}

      \begin{block}{Results}
        The full output is far too verbose, so I present what \Gephi
        gives as output for comparison with the `official' network on
        the bottom left. 4:31;26--4:56;30 -- +30 for co-
      \end{block}

      \begin{block}{Discussion} 
        The meaning of results, the future of the topic, error covering
      \end{block}

      \begin{block}{Conclusions}
        a re-hash of the second part of discussion?
      \end{block}

      \begin{block}{References}
        \centering
        \begin{minipage}{.9\linewidth}
          \printbibliography
        \end{minipage}
      \end{block}
    \end{column}
  \end{columns}
\end{frame}
\end{document}

%%% Local Variables:
%%% TeX-master: t
%%% TeX-PDF-mode: t
%%% End: